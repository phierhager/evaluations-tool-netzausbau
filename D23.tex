\documentclass[
    a4paper,
    11pt,
    parskip=half,   % Europäischer Absatzabstand statt Einzug
    ngerman         % Neue deutsche Rechtschreibung
]{scrartcl}

% --- Pakete ---
\usepackage[utf8]{inputenc}
\usepackage[T1]{fontenc}
\usepackage{babel}
\usepackage{lmodern}            % Bessere Schriftart
\usepackage[margin=2.5cm]{geometry} % Ränder anpassen
\usepackage{graphicx}
\usepackage{xcolor}
\usepackage{tabularx}           % Tabellen mit automatischer Spaltenbreite
\usepackage{booktabs}           % Professionelle Tabellenlinien
\usepackage{amsmath}            % Für mathematische Formeln
\usepackage{listings}           % Für Code-Blöcke
\usepackage{hyperref}           % Verlinkungen im PDF

% --- Farben definieren ---
\definecolor{hawblue}{RGB}{0, 51, 102} % Beispielhaftes Dunkelblau
\definecolor{codegray}{rgb}{0.95,0.95,0.95}
\definecolor{codecomment}{rgb}{0.0,0.6,0.0}
\definecolor{codestring}{rgb}{0.58,0.0,0.82}
\definecolor{codekeyword}{rgb}{0.0,0.0,1.0}

% --- Code-Style Einstellungen ---
\lstset{
    backgroundcolor=\color{codegray},
    commentstyle=\color{codecomment},
    keywordstyle=\color{codekeyword}\bfseries,
    stringstyle=\color{codestring},
    basicstyle=\ttfamily\footnotesize,
    breakatwhitespace=false,
    breaklines=true,
    captionpos=b,
    keepspaces=true,
    numbers=left,
    numbersep=5pt,
    numberstyle=\tiny\color{gray},
    showspaces=false,
    showstringspaces=false,
    showtabs=false,
    tabsize=2,
    frame=single,
    inputencoding=utf8,
    extendedchars=true,
    literate={ä}{{\"a}}1 {ö}{{\"o}}1 {ü}{{\"u}}1 {Ä}{{\"A}}1 {Ö}{{\"O}}1 {Ü}{{\"U}}1 {ß}{{\ss}}1
}

% Definition für JSON-Highlighting (da listings kein natives JSON hat)
\lstdefinelanguage{json}{
    basicstyle=\ttfamily\footnotesize,
    numbers=left,
    numberstyle=\tiny\color{gray},
    stepnumber=1,
    numbersep=8pt,
    showstringspaces=false,
    breaklines=true,
    frame=lines,
    backgroundcolor=\color{codegray},
    literate=
     *{0}{{{\color{blue}0}}}{1}
      {1}{{{\color{blue}1}}}{1}
      {2}{{{\color{blue}2}}}{1}
      {3}{{{\color{blue}3}}}{1}
      {4}{{{\color{blue}4}}}{1}
      {5}{{{\color{blue}5}}}{1}
      {6}{{{\color{blue}6}}}{1}
      {7}{{{\color{blue}7}}}{1}
      {8}{{{\color{blue}8}}}{1}
      {9}{{{\color{blue}9}}}{1}
      {:}{{{\color{black}:}}}{1}
      {,}{{{\color{black},}}}{1}
      {\{}{{{\color{black}\{}}}{1}
      {\}}{{{\color{black}\}}}}{1}
      {[}{{{\color{black}[}}}{1}
      {]}{{{\color{black}]}}}{1},
}

% --- Metadaten ---
\title{\textbf{Deliverable D2.3: Evaluations-Tool \& Maßnahmenkatalog}}
\subtitle{Projekt: PhyLFlex | Arbeitspaket 2.2}
\author{Verantwortlich: HAW Landshut (HAWL)}
\date{Version 2.0 (Überarbeitetes Konzept)}

% --- Dokumentenbeginn ---
\begin{document}

\maketitle

\begin{description}
    \item[Projekt:] PhyLFlex
    \item[Arbeitspaket:] AP 2.2 (Konzept eines Evaluations-Tools)
    \item[Partner:] TUM, ÜZW, SAG
\end{description}

\hrule
\tableofcontents
\vspace{1em}
\hrule
\vspace{1em}

% -------------------------------------------------------------------
\section{Management Summary (Zusammenfassung)}
Dieses Dokument definiert das Konzept für das \textbf{Evaluations-Tool}. Dieses Werkzeug fungiert als Wegweiser zwischen den konkurrierenden Interessen des Verteilnetzbetreibers (VNB/DSO) und den Energielieferanten/Verbrauchern.

Das Tool steuert das Netz nicht aktiv; es \textit{auditiert} die Netzsimulation. Es nimmt die physikalischen Ergebnisse (Spannungen, Lasten) entgegen und wendet eine rigorose \textbf{Kosten-Nutzen-Analyse} an, um festzustellen, ob eine spezifische Flexibilitätsmaßnahme im Vergleich zum klassischen Netzausbau wirtschaftlich rentabel ist.

% -------------------------------------------------------------------
\section{Maßnahmenkatalog}
Die folgenden spezifischen Interventionen sind für die Bewertung innerhalb des Tools definiert. Diese lassen sich direkt auf die im Förderantrag genannten regulatorischen Anforderungen abbilden.

\subsection{Kategorie A: Marktseitige Maßnahmen (Die „Ursache“)}
Diese Maßnahmen stellen Anreize dar, die an die Verbraucher gesendet werden, um deren Kosten zu optimieren.

\begin{itemize}
    \item \textbf{Maßnahme A1: Dynamische Tarife (§41a EnWG):} Das GEMS empfängt einen Preisvektor, der vom EPEX-Spotmarkt abgeleitet ist. Dies bedeutet primär Day-Ahead- und zeitvariable Preisgestaltung (Time of Use).
    \item \textbf{Maßnahme A2: Variable Netzentgelte (§14a Modul 3):} Reduzierte Netzentgelte im Austausch für die Duldung externer Steuerungseingriffe.
\end{itemize}

\subsection{Kategorie B: Netzseitige Maßnahmen (Die „Lösung“)}
Diese Maßnahmen sind die Reaktion des VNB auf den durch die Verbraucher verursachten Netzstress.

\begin{itemize}
    \item \textbf{Maßnahme B1: Kuratives Eingreifen (Dimmen):} Die temporäre Leistungsreduzierung von steuerbaren Verbrauchseinrichtungen (SteuVE), um eine unmittelbare Überlastung zu verhindern.
    \item \textbf{Maßnahme B2: Konventionelle Verstärkung:} Der physikalische Austausch von Betriebsmitteln (z.\,B. Ersatz eines 400 kVA Transformators durch eine 630 kVA Einheit oder die Verlegung von Parallelkabeln).
\end{itemize}

% -------------------------------------------------------------------
\section{Bewertungslogik: Das Kostenbilanzmodell}
Die Kernfunktion des Bewertungstools ist die Berechnung der \textbf{Systemgesamtkosten} für jedes Szenario. Es vergleicht die „Eisen-Investition“ (CAPEX) mit der „Flexibilitätsvergütung“ (OPEX).

\subsection{Kostenkomponente: Virtueller CAPEX (Netzausbau)}
Wenn die Simulation Spannungsverletzungen oder thermische Überlastungen aufzeigt, die nicht softwareseitig gelöst werden können, berechnet das Tool die Kosten des „Nichtstuns“ (physikalische Verstärkung):

\textbf{Auslöser (Triggers):}
\begin{itemize}
    \item \textbf{Unterspannung:} $< 207\,\text{V}$ (Netz zu schwach für die Last)
    \item \textbf{Überspannung:} $> 253\,\text{V}$ (Netz zu schwach für PV-Einspeisung)
    \item \textbf{Thermische Überlast:} $> 100\,\%$ der Kabel-/Trafo-Grenzwerte
\end{itemize}

\textbf{Formel:}
\[
\text{CAPEX} = (\text{Länge des ausgefallenen Kabels [km]} \times \text{Kosten [€/km]}) + \text{Kosten Trafoaustausch [€]}
\]

\subsection{Kostenkomponente: Interventions-OPEX (Abregelung)}
Wenn das Netz stabil bleibt, weil das GEMS die Leistungsflüsse aktiv gesteuert hat, berechnen wir den wirtschaftlichen Verlust, der mit diesen Eingriffen verbunden ist:

\begin{itemize}
    \item \textbf{A) Kosten der Lastabregelung (Komfortverlust):}
    \begin{itemize}
        \item \textit{Szenario:} Nutzer möchte E-Auto laden, Netz begrenzt die Leistung.
        \item \textit{Formel:} $\text{Abgeregelte Energie (kWh)} \times \text{Unannehmlichkeitsstrafe (€/kWh)}$
    \end{itemize}
    
    \item \textbf{B) Kosten der PV-Abregelung (Entgangene Einnahmen):}
    \begin{itemize}
        \item \textit{Szenario:} PV-Erzeugung übersteigt Netzkapazität; Wechselrichter wird gedrosselt.
        \item \textit{Formel:} $\text{Verlorene Erzeugung (kWh)} \times \text{Einspeisetarif (€/kWh)}$
    \end{itemize}
\end{itemize}

% -------------------------------------------------------------------
\section{Schnittstellenspezifikation}
Um sicherzustellen, dass das Bewertungstool objektiv ist, benötigt es ein standardisiertes Eingabeformat aus der Netzsimulation (AP 3). Dies trennt die \textit{Physik} (Simulation) von der \textit{Ökonomie} (Bewertung).

Die Netzsimulation muss einen Datensatz exportieren, der für jeden 15-Minuten-Zeitschritt die Felder in Tabelle \ref{felder} enthält.

\begin{table}[h]
\label{felder}
\centering
\begin{tabularx}{\textwidth}{lXl}
\toprule
\textbf{Feldname} & \textbf{Beschreibung} & \textbf{Einheit} \\
\midrule
\textbf{Zeitstempel} & ISO 8601 Datum/Zeit-String. & Zeit \\
\textbf{Netzstatus} & Kategorisches Kennzeichen (OK / WARNUNG / KRITISCH). & String \\
\textbf{Min. Spannung} & Die niedrigste an einem Knoten im Cluster gemessene Spannung. & Volt (V) \\
\textbf{Max. Spannung} & Die höchste an einem Knoten im Cluster gemessene Spannung. & Volt (V) \\
\textbf{Trafo-Auslastung} & Die prozentuale Spitzenauslastung des lokalen Transformators. & Prozent (\%) \\
\textbf{Kabelausfälle} & Gesamtlänge der Kabel, die thermische Grenzwerte überschreiten. & km \\
\textbf{Abgeregelte Last} & Von flexiblen Lasten (EV/WP) angeforderte, aber vom Netz verweigerte Energie. & kWh \\
\textbf{Abgeregelte PV} & Potenzielle PV-Erzeugung, die durch Netzlimits (Überspannung/Sättigung) unterdrückt wurde. & kWh \\
\bottomrule
\end{tabularx}
\caption{Datenfelder der Simulationsschnittstelle}
\end{table}

% -------------------------------------------------------------------
\section{KPI-Definitionen \& Verletzungsgrenzwerte}
Um eine eindeutige Bewertung zu gewährleisten, wird das Kennzeichen „Netzstatus“ aus den folgenden gesetzlichen Grenzwerten (EN 50160) abgeleitet.

\begin{table}[h]
\centering
\begin{tabularx}{\textwidth}{lXll}
\toprule
\textbf{KPI} & \textbf{Standard / Norm} & \textbf{Min. Schwelle} & \textbf{Max. Schwelle} \\
\midrule
Spannungshöhe & DIN EN 50160 & $< 207,0\,\text{V}$ (-10\%) & $> 253,0\,\text{V}$ (+10\%) \\
Trafo-Belastung & IEC 60076-7 & N/A & $> 100\,\%$ \\
Thermische Kabellast & DIN VDE 0276 & N/A & $> 100\,\%$ \\
\bottomrule
\end{tabularx}
\caption{Grenzwerte für KPIs}
\end{table}

% -------------------------------------------------------------------
\section{Synthese: Die Entscheidungsmatrix}
Das Endergebnis von D2.3 ist eine Entscheidungsmatrix, die die Kernfrage des Projekts beantwortet:

\begin{itemize}
    \item \textbf{Fragestellung:} Spart das GEMS (AP 4/5/8) Geld im Vergleich zum traditionellen Kupferausbau?
    \item \textbf{Methode:}
    \begin{enumerate}
        \item Berechnung von \textbf{Kosten\_Klassisch} (Szenario A mit vollem Netzausbau-CAPEX).
        \item Berechnung von \textbf{Kosten\_Smart} (Szenario C/D mit GEMS + Abregelungs-OPEX).
    \end{enumerate}
    \item \textbf{Erfolgsindikator:}
\end{itemize}

\[
\text{Wenn } \textbf{Kosten\_Smart} < \textbf{Kosten\_Klassisch} \rightarrow \text{GEMS-Konzept wirtschaftlich rentabel.}
\]

\newpage
\appendix

% ===================================================================
% ANHANG A
% ===================================================================
\section{Technische Spezifikationen \& Schnittstellen}
\textbf{Dokument:} Technische Anlage zu D2.3 (Evaluations-Tool)

\subsection{Schnittstelle: Simulationsergebnisse (Input für AP 2.2)}
Diese Schnittstelle definiert das Datenformat, das die Netzsimulation (AP 3) bereitstellen muss. Es werden zwei Formate unterstützt: \textbf{JSON} (bevorzugt) und \textbf{CSV}.

\subsubsection{Dateiformat: JSON}
Dateinamenskonvention: \texttt{results\_[SzenarioID]\_[Zeitstempel].json}

\begin{lstlisting}[language=json, caption={Beispiel JSON Output}]
{
  "meta": {
    "projectId": "PhyLFlex",
    "simulationId": "SIM_SCENARIO_C_2024",
    "description": "Szenario mit 50% EV-Durchdringung und GEMS-Steuerung",
    "createdAt": "2024-10-15T14:30:00Z",
    "timeStepMinutes": 15
  },
  "timeSeries": [
    {
      "timestamp": "2024-06-01T12:00:00Z",
      "gridStatus": "OK",
      "minVoltageV": 228.5,
      "maxVoltageV": 235.1,
      "trafoLoadPct": 45.2,
      "cableFailuresKm": 0.0,
      "curtailedLoadKwh": 0.0,
      "curtailedPvKwh": 0.0
    },
    {
      "timestamp": "2024-06-01T12:30:00Z",
      "gridStatus": "CRITICAL",
      "minVoltageV": 206.5,
      "maxVoltageV": 254.2,
      "trafoLoadPct": 102.1,
      "cableFailuresKm": 0.45,
      "curtailedLoadKwh": 15.0,
      "curtailedPvKwh": 8.2
    }
  ]
}
\end{lstlisting}

\subsubsection{Dateiformat: CSV}
Dateinamenskonvention: \texttt{results\_[SzenarioID].csv}
\begin{itemize}
    \item Trennzeichen: Komma (\texttt{,})
    \item Dezimaltrenner: Punkt (\texttt{.})
    \item Kodierung: UTF-8
\end{itemize}

\begin{lstlisting}[caption={Beispiel CSV Output}]
timestamp,grid_status,min_voltage_v,max_voltage_v,trafo_load_pct,cable_failures_km,curtailed_load_kwh,curtailed_pv_kwh
2024-06-01T12:00:00Z,OK,228.5,235.1,45.2,0.0,0.0,0.0
2024-06-01T12:15:00Z,WARNING,215.0,251.8,88.5,0.0,2.5,0.0
2024-06-01T12:30:00Z,CRITICAL,206.5,254.2,102.1,0.45,15.0,8.2
\end{lstlisting}

\subsubsection{Datendefinitionen}
\begin{table}[h]
\centering
\begin{tabularx}{\textwidth}{l l l l X}
\toprule
\textbf{Feld (JSON)} & \textbf{CSV Header} & \textbf{Typ} & \textbf{Einh.} & \textbf{Beschreibung} \\
\midrule
\texttt{timestamp} & \texttt{timestamp} & String & ISO & Zeitstempel (UTC). \\
\texttt{gridStatus} & \texttt{grid\_status} & String & Enum & Aggregierter Status. \\
\texttt{minVoltageV} & \texttt{min\_voltage\_v} & Float & V & Tiefste Spg. (L-N). \\
\texttt{maxVoltageV} & \texttt{max\_voltage\_v} & Float & V & Höchste Spg. (L-N). \\
\texttt{trafoLoadPct} & \texttt{trafo\_load\_pct} & Float & \% & Auslastung Trafo. \\
\texttt{cableFailuresKm} & \texttt{cable\_failures\_km} & Float & km & Länge überlastete Kabel. \\
\texttt{curtailedLoadKwh} & \texttt{curtailed\_load\_kwh} & Float & kWh & Nicht gelieferte Energie. \\
\texttt{curtailedPvKwh} & \texttt{curtailed\_pv\_kwh} & Float & kWh & Abgeregelte PV-Energie. \\
\bottomrule
\end{tabularx}
\end{table}

\subsection{Konfiguration: Ökonomische Parameter (Input für AP 2.2)}
Diese Konfigurationsdatei steuert die Bewertungsparameter. Sie ermöglicht Sensitivitätsanalysen.

\subsubsection{Dateiformat: JSON}
Dateiname: \texttt{config.json}

\begin{lstlisting}[language=json, caption={config.json}]
{
  "meta": {
    "configName": "Basis-Szenario 2025",
    "version": "1.0",
    "editor": "HAW Landshut",
    "currency": "EUR"
  },
  "economics": {
    "capex": {
      "description": "Kostenparameter fuer Netzausbau (Hardware)",
      "cable_reinforcement_cost_per_km": 80000.0,
      "trafo_replacement_cost_unit": 25000.0,
      "lifespan_years": 40,
      "interest_rate_pct": 3.5
    },
    "opex": {
      "description": "Kostenparameter fuer betriebliche Eingriffe",
      "pv_feed_in_tariff_eur_kwh": 0.082,
      "penalties": {
        "load_curtailment_penalty_eur_kwh": 1.50,
        "pv_curtailment_compensation_eur_kwh": 0.082
      }
    }
  }
}
\end{lstlisting}

\subsubsection{Parameter-Erklärung}
\begin{table}[h]
\centering
\begin{tabularx}{\textwidth}{l l X}
\toprule
\textbf{Parameter} & \textbf{Einh.} & \textbf{Beschreibung} \\
\midrule
\texttt{cable\_reinforcement...} & €/km & Kosten Tiefbau u. Material (NS-Kabel). \\
\texttt{trafo\_replacement...} & €/Stk & Pauschalkosten Trafo-Austausch. \\
\texttt{pv\_feed\_in\_tariff...} & €/kWh & Einspeisevergütung (EEG). \\
\texttt{load\_curtailment...} & €/kWh & Monetärer Wert für Komfortverlust. \\
\bottomrule
\end{tabularx}
\end{table}

\newpage

% ===================================================================
% ANHANG B
% ===================================================================
\section{Referenz-Implementierung (Python-Logik)}
\textbf{Dokument:} Technische Anlage zu D2.3 (Evaluations-Tool)

Dieser Anhang beschreibt die Referenz-Implementierung des \textbf{Cost Balancing Models} in Python.

\subsection{Voraussetzungen \& Bibliotheken}
Benötigte Bibliotheken:
\begin{itemize}
    \item \texttt{pandas}: Für die Verarbeitung der Zeitreihen.
    \item \texttt{json}: Für das Einlesen der Konfiguration.
\end{itemize}

\subsection{Implementierung: \texttt{GridEconomicEvaluator}}

\begin{lstlisting}[language=Python, caption={Evaluator Klasse in Python}]
import json
import pandas as pd

class GridEconomicEvaluator:
    """
    Klasse zur wirtschaftlichen Bewertung von Netzsimulationsdaten.
    Vergleicht 'Kupferausbau' (CAPEX) mit 'Flexibilitaet' (OPEX).
    """

    def __init__(self, config_path: str):
        # 1. Laden der Konfigurationsdatei (Preise & Grenzwerte)
        with open(config_path, 'r', encoding='utf-8') as f:
            self.config = json.load(f)
        
        # Extrahieren der Wirtschaftsparameter
        self.econ = self.config['economics']
        
    def evaluate_scenario(self, simulation_results_csv: str):
        """
        Fuehrt die Kostenanalyse fuer ein gegebenes Szenario durch.
        """
        # 2. Laden der Simulationsdaten
        df = pd.read_csv(simulation_results_csv)
        
        # --- A. Berechnung CAPEX (Virtueller Netzausbau) ---
        # Logik: Wenn Kabel thermisch ueberlastet sind, muessen sie
        # verstaerkt werden. Wir nehmen das Maximum an.
        
        max_cable_failure_km = df['cable_failures_km'].max()
        trafo_overload_events = df[df['trafo_load_pct'] > 100].shape[0]
        
        # Kostenberechnung Kabel
        cost_cabling = (max_cable_failure_km * self.econ['capex']['cable_reinforcement_cost_per_km'])
        
        # Kostenberechnung Trafo (Pauschal bei Ueberlastung)
        cost_trafo = 0.0
        if trafo_overload_events > 0:
            cost_trafo = self.econ['capex']['trafo_replacement_cost_unit']
            
        total_capex = cost_cabling + cost_trafo

        # --- B. Berechnung OPEX (Interventionskosten) ---
        # Logik: Summe der Energieverluste * Kosten.
        
        # Kosten durch PV-Abregelung
        total_curtailed_pv_kwh = df['curtailed_pv_kwh'].sum()
        cost_opex_pv = (total_curtailed_pv_kwh * self.econ['opex']['pv_feed_in_tariff_eur_kwh'])
        
        # Kosten durch Last-Verschiebung
        total_curtailed_load_kwh = df['curtailed_load_kwh'].sum()
        cost_opex_load = (total_curtailed_load_kwh * self.econ['opex']['penalties']['load_curtailment_penalty_eur_kwh'])
        
        total_opex = cost_opex_pv + cost_opex_load
        
        return {
            "capex_eur": round(total_capex, 2),
            "opex_eur": round(total_opex, 2),
            "total_system_cost_eur": round(total_capex + total_opex, 2),
            "details": {
                "cable_km_replaced": max_cable_failure_km,
                "trafo_replaced": trafo_overload_events > 0
            }
        }

# --- Anwendungsbeispiel (Main) ---
if __name__ == "__main__":
    evaluator = GridEconomicEvaluator("config.json")
    result = evaluator.evaluate_scenario("results_SIM_SCENARIO_C.csv")
    
    print("--- EVALUATION REPORT: PhyLFlex D2.3 ---")
    print(f"Szenario Kosten: {result['total_system_cost_eur']} EUR")
    
    if result['capex_eur'] > 0 and result['opex_eur'] < result['capex_eur']:
        print("FAZIT: GEMS-Einsatz ist wirtschaftlicher.")
    elif result['capex_eur'] == 0:
        print("FAZIT: Netz ist stabil.")
    else:
        print("FAZIT: Netzausbau waere guenstiger.")
\end{lstlisting}

\subsection{Integrationshinweise}
\begin{enumerate}
    \item \textbf{Zeitschritt-Skalierung:} Das Output-Format liefert Energiewerte (\texttt{kWh}). Falls die Simulation nur Leistung (\texttt{kW}) liefert, muss umgerechnet werden (bei 15-Min-Schritten: $\frac{kW}{4}$).
    \item \textbf{Kupferpreis-Index:} Der Parameter \texttt{cable\_reinforcement\_cost\_per\_km} sollte idealerweise jährlich aktualisiert werden.
    \item \textbf{Wiederholte Kabel-Fehler:} Die Logik \texttt{max()} für \texttt{cable\_failures\_km} verhindert, dass derselbe überlastete Kabelstrang mehrfach berechnet wird.
\end{enumerate}

\end{document}